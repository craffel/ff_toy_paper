\documentclass{article} % For LaTeX2e
\usepackage{iclr2016_workshop,times}
\usepackage{hyperref}
\usepackage{url}
\usepackage{amssymb}
\usepackage{amsmath}
\usepackage{booktabs}
\usepackage{graphicx}

\title{Feed-Forward Networks with Attention Can Solve Some Long-Term Memory Problems}


\author{Colin Raffel\\
LabROSA, Columbia University\\
\texttt{craffel@gmail.com}
\And
Daniel P.~W.~Ellis\\
LabROSA, Columbia University\\
\texttt{dpwe@ee.columbia.edu}
}

\newcommand{\fix}{\marginpar{FIX}}
\newcommand{\new}{\marginpar{NEW}}

\begin{document}

\maketitle

\begin{abstract}
Recurrent neural networks (RNNs) have proven to be powerful models in problems involving sequential data.
Recently, RNNs have been augmented with ``attention'' mechanisms which allow the network to focus on different parts of an input sequence when computing their output.
We propose a simplified model of attention which is applicable to feed-forward neural networks and demonstrate that the resulting model can solve the synthetic ``addition'' and ``multiplication'' long-term memory problems for sequence lengths which are both longer and more widely varying than the best results attained with RNNs.
\end{abstract}

\section{Models for Sequential Data}

Many problems in machine learning are best formulated using sequential data (i.e.\ data where a given observation may be dependent on previous observations), such as sequence transduction (producing a new sequence given an input sequence), sequence classification (producing a single label from an entire sequence), or sequence embedding (producing a single vector from an entire sequence).
Appropriate models for these tasks must be able to capture temporal dependencies in sequences, potentially of arbitrary length.
One such class of models are recurrent neural networks (RNNs), which can be considered a learnable function $f$ whose output $h_t = f(x_t, h_{t - 1})$ at time $t$ depends on input $x_t$ and the model's previous state $h_{t - 1}$.
Training of RNNs with backpropagation is hindered by the vanishing and exploding gradient problem \cite{pascanu2012difficulty,hochreiter1997long,bengio1994learning}, and as a result RNNs are in practice typically only applied in tasks where sequential dependencies span at most hundreds of time steps.
Very long sequences can also make training computationally inefficient due to the fact that RNNs must be evaluated sequentially and cannot be fully parallelized.

\subsection{Attention}

A recently proposed method for easier modeling of long-term dependencies is ``attention''.
Attention mechanisms allow for a more direct dependence between the state of the model at different points in time.
Following the definition from \cite{bahdanau2014neural}, given a model which produces a hidden state $h_t$ at each time step, attention-based models compute a ``context'' vector $c_t$ as the weighted mean of the state sequence $h$ by
$$
c_t = \sum_{j = 1}^T \alpha_{tj} h_j
$$
where $T$ is the total number of time steps in the input sequence and $\alpha_{tj}$ is a weight computed at each time step $t$ for each state $h_j$.
These context vectors are then used to compute a new state sequence $s$, where $s_t$ depends on $s_{t - 1}$, $c_t$ and, for sequence transduction, the model's output at $t - 1$.
The weightings $\alpha_{ij}$ are then computed by
\begin{equation*}
e_{tj} = a(s_{t - 1}, h_j), \alpha_{tj} = \frac{\exp(e_{tj})}{\sum_{k = 1}^T \exp(e_{tk})}
\end{equation*}
where $a$ is a learned function which can be thought of as computing a scalar importance value for $h_j$ given the value of $h_j$ and the previous state $s_{t - 1}$.
This formulation allows the new state sequence $s$ to have more direct access to the entire state sequence $h$.

\subsection{Feed-Forward Attention}

A straightforward simplification to the attention mechanism described above which would allow it to be applied to sequence embedding tasks could be formulated as follows:
Instead of a sequence of context vectors, we produce a single context vector $c$ as
\begin{equation}
\label{eq:ffattention}
e_t = a(h_t), \alpha_t = \frac{\exp(e_t)}{\sum_{k = 1}^T \exp(e_k)}, c = \sum_{t = 1}^T \alpha_t h_t
\end{equation}
As before, $a$ is a learnable function, but it now only depends on $h_t$.
In this formulation, attention can be seen as producing a fixed-length embedding $c$ of the input sequence by computing an adaptive weighted average of the state sequence $h$.
\cite{sonderby2015convolutional} compared the effectiveness of a standard recurrent network to a recurrent network augmented with this simplified version of attention on the task of protein sequence analysis.

A consequence of using an attention mechanism is the ability to integrate information over time.
It follows that by using this simplified form of attention, a model could perform sequence embedding even if the calculation of $h_t$ was feed-forward, i.e.\ $h_t = f(x_t)$.
Using a feed-forward $f$ could also result in large efficiency gains as the computation could be completely parallelized.

We note here that feed-forward models without attention can be used for sequence embedding when the sequence length $T$ is fixed, but when $T$ varies across sequences, some form of temporal integration is necessary.
An obvious straightforward choice, which can be seen as an extreme oversimplification of attention, would be to compute $c$ as the unweighted average of the state sequence $h_t$, i.e.
\begin{equation}
\label{eq:unweighted}
c = \frac{1}{T}\sum_{t = 1}^T h_t
\end{equation}
We will also explore the effectiveness of this approach for sequence embedding.

\section{Toy Long-Term Memory Problems}

A common way to measure the long-term memory capabilities of a given model is to test it on the synthetic problems originally proposed in \cite{hochreiter1997long}.
In this paper, we will focus on the ``addition'' and ``multiplication'' problems; due to space constraints, we refer the reader to \cite{hochreiter1997long} or \cite{sutskever2013importance} for their specification.
All of the code used in our experiments is available online.\footnote{\href{https://github.com/craffel/ff-attention/tree/master/toy_problems}{\texttt{https://github.com/craffel/ff-attention/tree/master/toy\char`_problems}}}

\subsection{Model Details}

For all experiments, we used the following model:
First, the state $h_t$ was computed from the input at each time step $x_t$ by $h_t = \textrm{LReLU}(W_{xh}x_t + b_{xh})$ where $W_{xh} \in \mathbb{R}^{D \times 2}, b_{xh} \in \mathbb{R}^D$ and $\textrm{LReLU}(x) = \max(x, .01x)$ is the ``leaky rectifier'' nonlinearity, as proposed in \cite{maas2013rectifier}.
We found that this nonlinearity improved early convergence so we used it in all of our models.
We tested models where the context vector $c$ was then computed either as in Equation \ref{eq:ffattention}, with $a(h_t) =\tanh(W_{hc}h_t + b_{hc})$
where $W_{hc} \in \mathbb{R}^{1 \times D}, b_{hc} \in \mathbb{R}$, or simply as the unweighted mean of $h$ as in Equation \ref{eq:unweighted}.
We then computed an intermediate vector $s = \textrm{LReLU}(W_{cs}c + b_{cs})$ where $W_{cs} \in \mathbb{R}^{D \times D}, b \in \mathbb{R}^D$ from which the output was computed as $y = \textrm{LReLU}(W_{sy}s + b_{sy})$ where $W_{sy} \in \mathbb{R}^{1 \times D}$, $b_{sy} \in \mathbb{R}$.
For all experiments, we set $D = 100$.

We used the squared error of the output $y$ against the target value for each sequence as an objective.
Parameters were optimized using ``adam'', a recently proposed stochastic optimization technique \cite{kingma2014adam}, with the optimization hyperparameters $\beta_1$ and $\beta_2$ set to the values suggested in \cite{kingma2014adam} (.9 and .999 respectively).
All weight matrices were initialized with entries drawn from a Gaussian distribution with a mean of zero and, for a matrix $W \in \mathbb{R}^{M \times N}$, a standard deviation of $1/\sqrt{N}$.
All bias vectors were initialized with zeros.
We trained on mini-batches of 100 sequences and computed the accuracy on a held-out test set of 1000 sequences every epoch, defined as 1000 parameter updates.
We stopped training when either 100\% accuracy was attained on the test set, or after 100 epochs.
All networks were implemented using Lasagne \cite{dieleman2015lasagne}, which is built on top of Theano \cite{bastien2012theano,bergstra2010theano}.

\begin{table}
  \centering
  \footnotesize
  \begin{tabular}{r c c c c c c c c c c c c}
    \toprule
    Task & \multicolumn{6}{c}{\textbf{Addition}} & \multicolumn{6}{c}{\textbf{Multiplication}} \\
    $T_0$ & 50 & 100 & 500 & 1000 & 5000 & 10000 & 50 & 100 & 500 & 1000 & 5000 & 10000 \\
    \cmidrule(r){2-7}
    \cmidrule(r){8-13}
    Attention & 1 & 1 & 1 & 1 & 2 & 3 & 1 & 2 & 4 & 2 & 15 & 6 \\
    Unweighted & 1 & 1 & 1 & 2 & 8 & 17 & 2 & 2 & 8 & 33 & \textcolor{gray}{89.8\%} & \textcolor{gray}{80.8\%} \\
    \bottomrule
  \end{tabular}
  \caption{Number of epochs required to achieve perfect accuracy, or accuracy after 100 epochs (greyed-out values), for the experiment described in Section \ref{sec:fixed}.}
  \label{tab:fixed}
\end{table}

\subsection{Fixed-Length Experiment}
\label{sec:fixed}

Traditionally, the sequence lengths tested in each task vary uniformly between $[T_0, 1.1T_0]$ for different values of $T_0$.
As $T_0$ increases, the model must be able to handle longer-term dependencies.
The largest value of $T_0$ attained using RNNs with different training, regularization, and model structures has varied from a few hundred \cite{martens2011learning,sutskever2013importance,le2015simple,krueger2015regularizing,arjovsky2015unitary} to a few thousand \cite{hochreiter1997long,jaegar2012long}.
We therefore tested our proposed feed-forward attention models for $T_0 \in \{50, 100, 500, 1000, 5000, 10000\}$.
The required number of epochs or accuracy after 100 epochs for each task, sequence length, and temporal integration method (adaptively weighted attention or unweighted mean) is shown in Table \ref{tab:fixed}.
For fair comparison, we report the best result achieved using any learning rate in $\{.0003, .001, .003, .01\}$.
From these results, it's clear that the feed-forward attention model can quickly solve these long-term memory problems for longer sequence lengths than have been demonstrated with RNNs.
Our model is also efficient: Processing one epoch of 100,000 sequences with $T_0 = 10000$ took 254 seconds using an NVIDIA GTX 980 Ti GPU, while processing the same data with a single-layer vanilla RNN with a hidden dimensionality of 100 (resulting in a comparable number of parameters) took 917 seconds on the same hardware.
In addition, there is a clear benefit to using the attention mechanism of Equation \ref{eq:ffattention} instead of a simple unweighted average over time, which only incurs a marginal increase in the number of parameters (10,602 vs.\ 10,501, or less than 1\%).

\subsection{Variable-length Experiment}

Because the range of sequence lengths $[T_0, 1.1T_0]$ is small compared to the range of $T_0$ values we evaluated, we further tested whether it was possible to train a single model which could cope with sequences with highly varying lengths.
To our knowledge, such an experiment has not been conducted with RNNs.
We trained models of the same architecture as used in the previous experiment on minibatches of sequences whose lengths were chosen uniformly at random between 50 and 10000 time steps.
Using the attention mechanism of Equation \ref{eq:ffattention}, on held-out test sets of 1000 sequences, our model achieved 99.9\% accuracy on the addition task and 99.4\% on the multiplication task after training for 100 epochs.
This suggests that a single feed-forward network with attention can simultaneously model both short-term and very long-term dependencies, with a marginal decrease in accuracy.
Using an unweighted average over time, we were only able to achieve accuracies of 77.4\% and 55.5\% on the variable-length addition and multiplication tasks, respectively.

These results clearly show that our model can be substantially more efficient than recurrent models on some classic synthetic long-term memory tasks.
A clear limitation of our proposed model is that it ignores temporal order because integrating over time discards order information.
Nevertheless, we submit that in many cases, temporal order is substantially less important than being able to handle very long sequences.
In fact, we have shown in parallel work that our proposed feed-forward attention model can be used effectively for pruning large-scale (sub)sequence retrieval searches, even when the sequences are very long and high-dimensional \cite{raffel2016pruning}.
We are optimistic that this model will prove beneficial in additional real-world problems requiring order-agnostic temporal integration of long sequences; further investigation is warranted.

\bibliography{refs}
\bibliographystyle{iclr2016_workshop}

\end{document}
