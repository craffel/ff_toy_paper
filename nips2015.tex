\documentclass{article} % For LaTeX2e
\usepackage{nips15submit_e,times}
\usepackage{hyperref}
\usepackage{url}
%\documentstyle[nips14submit_09,times,art10]{article} % For LaTeX 2.09


\title{Feed-Forward Networks with Attention Can Solve Some Long-Term Memory Problems}


\author{
Colin Raffel
LabROSA
Columbia University
New York, NY 10027
\texttt{craffel@gmail.com}
\And
Daniel P.~W.~Ellis
LabROSA
Columbia University
New York, NY 10027
\texttt{dpwe@ee.columbia.edu}
}

\newcommand{\fix}{\marginpar{FIX}}
\newcommand{\new}{\marginpar{NEW}}

%\nipsfinalcopy % Uncomment for camera-ready version

\begin{document}

\maketitle

\begin{abstract}
Recently, recurrent neural networks (RNNs) have been augmented with ``attention'' mechanisms which compute a fixed-length representation of entire sequences.
We propose a simplified model of attention which is applicable to feed-forward neural networks and demonstrate that it can solve some long-term memory problems (specifically, those where temporal order doesn't matter).
In fact, we show empirically that our model can solve these problems for sequence lengths which are both longer and more widely varying than has been shown for RNNs.
\end{abstract}

\section{Models for Sequential Data}

Many problems in machine learning are best formulated using sequential data, i.e.\ data where a given observation may be dependent on previous observations.
Such problems can be coarsely classified as sequence transduction (producing a new sequence given an input sequence), sequence embedding (producing a single label, value, or vector from an entire sequence), or sequence generation (producing a sequence from no input) tasks.
Appropriate models for these tasks must be able to capture temporal dependencies in sequences, potentially of arbitrary length.

\subsection{Recurrent Neural Networks}

One such class of models are recurrent neural networks (RNNs), which can be considered as a learnable function $f$ whose output $h_t = f(x_t, h_{t - 1})$ at time $t$ depends on input $x_t$ and the previous state $h_{t - 1}$.
In the supervised setting, the parameters of $f$ are optimized with respect to a loss function which measures $f$'s performance.
A common approach is to use backpropagation through time \cite{werbos1990backpropagation}, which ``unrolls'' the RNN over time steps to compute the gradient of the parameters of $f$ with respect to the loss.
Because the same function $f$ is applied repeatedly over time, this gradient can easily explode or vanish \cite{pascanu2012difficulty,hochreiter1997long,bengio1994learning}.
The use of gating architectures \cite{hochreiter1997long,cho2014learning}, sophisticated optimization techniques \cite{martens2011learning,sutskever2013importance}, gradient clipping \cite{pascanu2012difficulty,graves2013generating}, and/or careful initialization \cite{sutskever2013importance,jaegar2012long,mikolov2014learning,le2015simple} can help mitigate this issue and has facilitated the success of RNNs in a variety of fields (see e.g.\ \cite{graves2012supervised,cho2015describing} for an overview).
However, these approaches don't \textit{solve} the problem of vanishing and exploding gradients, and as a result RNNs are in practice typically only applied in tasks where sequential dependencies span at most hundreds of time steps \cite{martens2011learning,sutskever2013importance,le2015simple,hochreiter1997long}.
Very long sequences can also make training computationally inefficient due to the fact that RNNs must be evaluated sequentially and cannot be fully parallelized.

\subsection{Attention}

A recently proposed method for easier modeling of long-term dependencies is ``attention''.
Attention mechanisms allow for a more direct dependence between the state of the model at different points in time.
Following the definition from \cite{bahdanau2014neural}, given a model which produces a hidden state $h_t$ at each time step, attention-based models first compute a ``context'' vector $c_t$ as the weighted mean of the state sequence $h$ by
$$
c_t = \sum_{j = 1}^T \alpha_{tj} h_j
$$
where $T$ is the total number of time steps in the input sequence and $\alpha_{tj}$ is a weight computed at each time step $t$ for each state $h_j$.
These context vectors are then used to compute a new state sequence $s$, where $s_t$ depends on $s_{t - 1}$, $c_t$ and, for sequence prediction, the model's output at $t - 1$.
The weightings $\alpha_{ij}$ are then computed by
$$
e_{tj} = a(s_{t - 1}, h_j),\; \alpha_{tj} = \frac{\exp(e_{tj})}{\sum_{k = 1}^T \exp(e_{tk})}
$$
where $a$ is a learned function which can be thought of as computing a scalar importance value for $h_j$ given the value of $h_j$ and the previous state $s_{t - 1}$.
This formulation allows the new state sequence $s$ to have more direct access to the entire state sequence $h$.
Attention-based RNNs have proven effective in a variety of sequence transduction tasks \cite{bahdanau2014neural,cho2015describing}.
RNNs with attention can be seen as analogous to the recently proposed Memory Network \cite{weston2014memory,sukhbaatar2015end} and Neural Turing Machine \cite{graves2014neural} models.

\subsection{Feed-Forward Attention}

A straightforward simplification to the attention mechanism described above which would allow it to be applied to sequence embedding tasks could be formulated as follows:
Instead of a sequence of context vectors, we produce a single context vector $c$ as
$$
c = \sum_{t = 1}^T \alpha_t h_t,\; e_t = a(h_j),\; \alpha_t = \frac{\exp(e_t)}{\sum_{k = 1}^T \exp(e_k)}
$$
As before, $a$ is a learnable function, but it now only depends on $h_j$.
In this formulation, attention can be seen as producing a fixed-length embedding $c$ of the input sequence by computing an adaptive weighted average of the state sequence $h$.

A consequence of using an attention mechanism is that the context vector $c$ can model temporal dependencies because attention performs integration over time.
It follows that by using this simplified form of attention, a model could perform sequence embedding even if the calculation of $h_t$ was feed-forward, i.e.\ $h_t = f(x_t)$.
While using a completely feed-forward model for sequential modeling tasks will sacrifice the ability to solve some problems, we show that for certain tasks, feed-forward networks with attention can perform arbitrary-length sequence embedding more effectively than RNNs.

We note here that feed-forward models can be used for sequence embedding when the sequence length $T$ is fixed, but when $T$ varies across sequences, some form of temporal integration is necessary.
An obvious straightforward choice, which can be seen as an extreme oversimplification of attention, would be to compute $c$ as the unweighted average of the state sequence $h_t$.
We will also explore the effectiveness of this approach for sequence embedding.

\section{Toy Long-Term Memory Problems}

A common way to measure the long-term memory capabilities of a given model is to test it on the synthetic problems originally proposed in \cite{hochreiter1997long}.
In this paper, we will focus on the adding and multiplication problems from \cite{hochreiter1997long} and the XOR problem from \cite{martens2011learning}.
These tasks can be summarized as follows:
The input is a two dimensional sequence, where one dimension is a random sequence (sampled uniformly from $[0, 1]$ in the addition and multiplication tasks and from $\{0, 1\}$ in the XOR task) and the other dimension is a ``mask'' sequence.
At two time steps, one in the first ten sequence steps and one before the sequence's midpoint, the ``mask'' signal is 1; at the first and last time step it is -1; and at all other time steps it is 0.
The goal is to perform addition, multiplication, or exclusive-or on the two values in the noise dimension which co-occur with the 1s in the mask dimension, which is meant to require that a model be able to store the correct values for the duration of the sequence.
Slight variants of these tasks have also been used \cite{sutskever2013importance,le2015simple,jaegar2012long,martens2011learning}, but we will follow the original definition from \cite{hochreiter1997long}.
These three tasks are only a subset of the synthetic long-term memory problems which have been proposed; we focus on them here because they are the most commonly used and discuss the applicability of feed-forward attention on the remaining problems in Section \ref{sec:limitations}.



Previous results

\subsection{Model Details}

\subsection{Fixed-Length Experiment}

\subsection{Variable-length Experiment}

\section{Limitations}
\label{sec:limitations}

\bibliographystyle{unsrt}
\small
\bibliography{refs}

\end{document}
